\documentclass[12pt]{book}
\usepackage[margin=3cm]{geometry}

\usepackage{subfiles}
\usepackage{csquotes}
\usepackage{amsmath}
\usepackage{amsthm}
\usepackage{amssymb}
\usepackage{quiver}
\usepackage{enumitem}

\usepackage{titlesec}
\titleformat{\subsection}[runin]{\normalfont\bfseries}{\thesubsection}{1em}{}
\titleformat{\subsubsection}[runin]{\normalfont\bfseries}{\thesubsubsection}{1em}{}
\setcounter{secnumdepth}{5}

\newcommand{\rat}{\mathrm{rat}}
\newcommand{\alg}{\mathrm{alg}}
\newcommand{\num}{\mathrm{num}}
\renewcommand{\hom}{\mathrm{hom}}
\newcommand{\eff}{\mathrm{eff}}
\newcommand{\mot}{\mathrm{mot}}
\newcommand{\nil}{\mathrm{nil}}

\newcommand{\Fr}{\mathrm{Fr}}
\newcommand{\pr}{\mathrm{pr}}
\newcommand{\tr}{\mathrm{tr}}
\newcommand{\Tr}{\mathrm{Tr}}
\newcommand{\sgn}{\mathrm{sgn}}
\newcommand{\h}{\mathfrak{h}}
\newcommand{\et}{\mathrm{ét}}
\newcommand{\op}{\mathrm{op}}
\newcommand{\id}{\mathrm{id}}

\newcommand{\1}{\mathbf{1}}
\renewcommand{\P}{\mathbb{P}}
\newcommand{\F}{\mathbb{F}}
\newcommand{\Q}{\mathbf{Q}}
\newcommand{\E}{\mathcal{E}}
\newcommand{\C}{\mathbf{C}}
\newcommand{\Z}{\mathbf{Z}}

\DeclareMathOperator{\Gr}{\mathbb{G}r}
\DeclareMathOperator{\NM}{NM}
\DeclareMathOperator{\AM}{AM}
\DeclareMathOperator{\CH}{CH}
\DeclareMathOperator{\Spec}{Spec}
\DeclareMathOperator{\Hom}{Hom}
\DeclareMathOperator{\Aut}{Aut}
\DeclareMathOperator{\uAut}{\underline{Aut}}
\DeclareMathOperator{\End}{End}
\DeclareMathOperator{\Gal}{Gal}

\title{AN INTRODUCTION TO MOTIVES\\(PURE MOTIVES, MIXED MOTIVES, PERIODS)}
\author{Yves André\\Translated to English by E. Hecky}

\begin{document}
\maketitle

\paragraph*{Abstract}

\emph{Motives} have been introduced 40 years ago by A. Grothendieck as \enquote{a systematic theory of arithmetic properties of algebraic varieties as embodied in their groups of classes of cycles}. 
This text provides an exposition of the geometric foundations of the theory (pure and mixed), and a panorama of major developments which have occurred in the last 15 years.
The last part is devoted to a study of \emph{periods} of motives, with emphasis on examples (polyzeta numbers, notably).

\paragraph*{Foreword}

Many disciplines have adopted in their technical vocabulary the term \emph{motive}, with one of its standard meanings: that of \enquote{reason} (which has a subjective connotation) and that of \enquote{constitutive element}.

By introducing this term in Algebraic Geometry 40 years ago, A. Grothendieck was playing with both meanings at once.
The goal was to find the common \emph{motive} in different cohomological phenomena (for example, the transcendental and arithmetic notions of weight), or even the \emph{motive} explaining certain mysterious relations between the integrals of algebraic functions, and so on.
The aim was also to decompose algebraic varieties, from a cohomological point of view, in simple \emph{motives} prone to recombination.

The fundamental intuition of Grothendieck was that this chemistry of motives was ruled by the theory of algebraic correspondences.
Motives should form some sort of purely algebraic universal cohomology, wherefrom all the other cohomology theories could be deduced as different \enquote{realizations}.
They should also give birth to a generalization of Galois theory to systems of multivariate polynomials.\\



One can distinguish three eras in the evolution of the theory.
Up until the end of the 1970s, it was basically reduced to the conjectural grothendieckian corpus of pure motives, as it is exposed at the end of Saavedra's book \cite{saavedra72}.
This piece of \enquote{sci-fi mathematics} was only referred to as a source of inspiration.

In the 1980s the theory has seen two major developments.
The first is the finding of non-conjectural results, even if it required to sacrifice the very notion of motive and to replace it with realizations systems.
The second is the establishment of a wonderful program towards a general theory of mixed motives, and the key idea of motivic cohomology.
The Deligne, Beilinson and Block-Kato conjectures on the relations between the values of $L$-functions and regulators have played a major role in making the theory popular amongst the arithmeticians.
These developments are exposed in the collective work \emph{Motives} (Seattle Conference 1991, AMS).

With the focus shifting back to algebraic correspondences, the next years have seen genuine breakthroughs giving non-conjectural foundations to the theory, both in pure motives with the proof of the conjecture on the semi-simplicity of numerical motives, and in mixed motives with a mature theory of motivic cohomology\footnote{enough to initiate work on the Milnor-Bloch-Kato conjectures.}.

Our aim is to make these developments accessible to non-specialists, while giving them in the first two parts a unitary vision of the theory\footnote{restricting ourselves to the geometric and categorical aspects: there will be no treatment of $L$-functions and nothing will be said about \enquote{Langlands philosophy} \cite{langlands79}}.

\part{PURE MOTIVES}

\chapter{ORIGINS : ENUMERATIVE GEOMETRY, COHOMOLOGY, GALOIS THEORY}

\subfile{part1/chapter1_origins.tex}

\chapter{RIGID $\otimes$-CATEGORIES, TANNAKIAN CATEGORIES}

\subfile{part1/chapter2_tannakian.tex}

\chapter{ALGEBRAIC CYCLES AND COHOMOLOGIES\\(CASE OF SMOOTH PROJECTIVE VARIETIES)}

\subfile{part1/chapter3_algebraiccycles.tex}

\end{document}