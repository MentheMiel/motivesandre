\documentclass[../main.tex]{subfiles}

\begin{document}
In this \enquote{heuristical} chapter, we will outline freely and without trying to be exhaustive, some common threads leading to motives.
It is not a historical reconstruction of the birth of the theory, and the notions will be reviewed in greater details in the upcoming chapters.

\section{Enumerative geometry}

\subsection{} One of the oldest and most elementary theorems in enumerative projective geometry is that of Bézout\footnote{1779, but stated by MacLaurin in 1720.} on the number of points of intersection between two plane curves without common component.
If the curves are defined as the vanishing loci of polynomials $P(x, y)$ and $Q(x, y)$ respectively, then they intersect at $\deg P \cdot \deg Q$ points, without forgetting multiplicities and points at infinity.
Poncelet\footnote{Traité des propriétés projectives des figures (1822).} suggested a \emph{dynamical} approach to this theorem, deforming $Q$ into a product $\prod (a_ix + b_iy)$ of linear components in general position and using his \enquote{conservation of number principle}.
It is then clear that each of the $\deg Q$ lines $a_ix + b_iy$ intersects $\{P = 0\}$ in exactly $\deg P$ points, and the theorem follows.

A modern formulation of this principle is the fact that the algebraic equivalence of algebraic cycles, and \emph{a fortiori} their rational equivalence, is finer than the numerical equivalence. Let us briefly recall their definitions.

An \emph{algebraic cycle} on a smooth projective variety $X$ is a formal linear combination $Z = \sum n_i Z_i$ of irreducible subvarieties $Z_i$ of $X$, with integers coefficients $n_i$ (or rational, depending on the context). Two cycles $Z$ and $Z'$ are said to be \emph{rationally equivalent} (denoted $Z \sim_{\rat} Z'$) if they may be transformed into one another by a succession of rational deformations, parametrized by the projective line $\P^1$. They are said to be \emph{algebraically equivalent} (denoted $Z \sim_{\alg} Z'$) if they can be transformed into one another by algebraic deformations. They are said to be \emph{numerically equivalent} (denoted $Z \sim_{\num} Z'$) if the intersection numbers $Z \cdot Y = \sum n_i Z_i \cdot Y$ and $Z' \cdot Y = \sum n'_i Z'_i \cdot Y$ agree for every subvariety $Y$ of $X$ of complementary codimension.

Obviously, $\sim_{\rat} \implies \sim_{\alg}$, and the \enquote{conservation of number principle} of Poncelet states that $\sim_{\alg} \implies \sim_{\num}$.

The theory of these equivalences on algebraic cycles had already developed around 1925, especially thanks to efforts by Van der Waerden and the Italian School (independently).

\paragraph*{Basic examples}
\begin{enumerate}
    \item A \emph{divisor} is an algebraic cycle $\sum n_i Z_i$ where each $Z_i$ is of codimension one (and each $n_i$ is an integer).
    One can rationally deform such a divisor to one with nonnegative coefficients, in its own linear system as long as the latter is nonempty.

    \item Let $Z = \sum n_i Z_i$ be a divisor on a curve $X$.
    Then $Z \sim_{\num} 0 \iff Z \sim_{\alg} 0 \iff \sum n_i = 0$.
    If $X$ is an elliptic curve, then $Z \sim_{\rat} 0$ if and only if $\sum n_i = 0$ \emph{and} $\sum_X n_i Z_i = 0_X$, where $\sum_X$ is the group law on the elliptic curve $X$.
    In higher genus, a similar criterion holds, substituting $X$ for its associated abelian \enquote{jacobian} variety.
\end{enumerate}

\subsection{} Enumerative geometry traditionally deals with counting geometric objects in a given family, having prescribed relations with a given configuration (Chasles, Schubert\footnote{Kalkül der abzählenden Geometrie (1879).}, Zeuthen\footnote{Lehrbuch der abzählenden Methoden der Geometrie (1914).}, ...).
It proceeds by rephrasing the problem in terms of intersection numbers of subvarieties of grassmannians parametrizing the configurations. These intersection numbers are the central subject of Schubert calculus.

\paragraph*{Basic examples}
\begin{enumerate}
    \item Counting the number of lines in $\P^3$ intersecting four given lines in general position. We find two: the lines intersecting a given line in $\P^3$ are parametrized by a divisor $\sigma_1$ in the 4-dimensional grassmannian $\Gr(1, 3)$.
    According to Schubert calculus, the self-intersection number $\sigma_1^4$ is $2$.
    In fact, this is another instance of the \enquote{conservation of number principle}\footnote{as J.-B. Bost pointed out.}: with deformations, one reduces the problem to the case where the first two (resp. last two) lines lie in a plane $P$ (resp. $Q$), intersect in a point $A$ (resp. $B$), and are in general position for those constraints.
    The two lines are then $AB$ and $P \cap Q$.
    
    \item Another classical counting problem in algebraic geometry is that of counting solutions to polynomial equations in finite fields (Gauss, Artin, Schmidt, Hasse, Weil, ...). Let $\nu_n$ be the number of points in $\F_{q^n}$ of a given smooth projective variety $X$ defined over $\F_q$. We encode these numbers in the zeta function $Z_X$, defined by the series
    $$\log Z_X(t) = \sum \nu_n \frac{t^n}{n}.$$
    One can compute them as intersection numbers on $X \times X$ :
    $$\nu_n = \langle \Delta, \Gamma_{\Fr_X^n} \rangle$$
    where $\Delta$ is the diagonal subvariety in $X \times X$, and $\Gamma_{\Fr_X^n}$ is the graph of the $n$-fold iterated Frobenius map of $X$ (locally, $\Fr_X$ is induced by $x \mapsto x^q$).
    As for many other enumerative problems on curves, one reduces it to studying algebraic correspondences modulo numerical equivalence between curves.
\end{enumerate}

\subsection{} All of these enumerative problems live naturally in the \enquote{enumerative category of smooth projective varieties} over a field $k$.

This $\Q$-linear category, temporarily denoted $\E(k)$, has as objects the smooth projective varieties on $k$.
The collection of morphisms is given by the degree $0$ algebraic correspondences modulo numerical equivalence
\begin{align*}
    \E(k)(X, Y) = \{\text{algebraic cycles }&\text{with rational coefficients}\\
    &\text{of codimension }\dim X\text{ on }X \times Y\} / \sim_{\num}.
\end{align*}
The composition of morphisms is given by the well-known rule from intersection theory :
$$g \circ f = (\pr^{XYZ}_{XZ})_* \left((\pr^{XYZ}_{XY})^* f \cdot (\pr^{XYZ}_{YZ})^* g\right)$$
involving the three \enquote{3-to-2} projections from $X \times Y \times Z$.

One could say that enumerative geometry in the broadest sense is the study of the category $\E(k)$.
It seems that A. Grothendieck was the first one to consider this category in and of itself, and to point out its remarkable properties.

These properties are even more apparent if one takes the \emph{pseudo-abelian envelope} of $\E(k)$, obtained by formally introducing kernels of idempotent endomorphisms.
In Grothendieck's lingo, this pseudo-abelian envelope denoted $\NM^{\eff}(k)_{\Q}$ is called the \emph{category of effective numerical motives}.
Grothendieck had conjectured the following \enquote{miraculous} property :
$$\NM^{\eff}(k)_{\Q} \text{ is a semi-simple abelian category}$$
which was later proved by U. Jannsen in 1991 \cite{jannsen92}.
The category $\NM^{\eff}(k)_{\Q}$ is moreover endowed with a monoidal structure induced by the cartesian product of varieties.

\subsubsection{Recall.} An additive category $\mathcal{C}$ is said to be \emph{pseudo-abelian} if for each endomorphism $e \in \mathcal{C}(A, A)$ satisfying $e^2 = e$, one can write $A$ as a direct sum $A_1 \oplus A_2$ such that $e$ is the projection $A \to A_1$ followed by the inclusion $A_1 \to A$.
We then say that $A_1$ is the image of $e$ and denote it $e(A)$ or $eA$. We observe that $A_2$ is then the image of the idempotent endomorphism $1 - e$.

Let $\mathcal{C}$ be an additive category.
Let $\mathcal{C}^{\natural}$ be the category whose objects are the pairs $(A, e)$ with $A$ an object of $\mathcal{C}$ and $e \in \mathcal{C}(A, A)$ an idempotent endomorphism.
The abelian group of morphisms from $(A, e)$ to $(A', e')$ is defined as $e' \circ \mathcal{C}(A, A') \circ e$, namely the subgroup of $\mathcal{C}(A, A')$ spanned by morphisms of the form $e' \circ f \circ e$.
One checks without effort that $\mathcal{C}^{\natural}$ is pseudo-abelian.
It is the \emph{pseudo-abelian envelope} of $\mathcal{C}$.
The obvious functor $\mathcal{C} \to \mathcal{C}^{\natural}$ is fully faithful.

\section{Cohomology of algebraic varieties}

\subsection{} When the base field is the field of complex numbers $\C$, the use of topological methods - especially cohomology theories - in algebraic geometry dates back to Poincaré, Picard, and then Lefschetz\footnote{the applications to the study of complex algebraic varieties was a major incentive for those authors.}.

Through various ways (simplicial methods, differential forms, ...), we assign to each smooth projective variety $X$ over a field $k$ a graded $k$-algebra $H^*(X)$.
It satisfies the \enquote{Künneth formula}: $H^*(X \times Y) \cong H^*(X) \otimes H^*(Y)$.
Moreover, we assign to each algebraic cycle of codimension $r$ over $X$ its fundamental class, which lives in $H^{2r}(X)$.
The \enquote{cup-product} of $H^*(X)$ is compatible with the intersection product of cycles.

Two cycles $Z$ and $Z'$ are said to be \emph{homologically equivalent} (denoted $Z \sim_{\hom} Z'$) if the fundamental class of their difference is zero.
This equivalence lies between the algebraic and the numerical ones.

\paragraph*{Basic example} Back to Bézout's theorem. We have $H^2(\P^2, \Q) = H^4(\P^2, \Q) = \Q$ and the cup-product $H^2(\P^2, \Q) \times H^2(\P^2, \Q) \to H^4(\P^2, \Q)$ is multiplication.
The fundamental class of a curve $C \subset \P^2$ is given by its degree and the fundamental class of a zero-dimensional cycle $\sum n_i P_i$ is its degree $\sum n_i$.
The number of points at the intersection of two curves $C$ and $C'$ intersecting properly is then given by the class of $C \cdot C'$, namely $\deg C \cdot \deg C'$.

The Lefschetz trace formula provides a powerful cohomological tool to compute the number of (isolated) fixed points of an endomorphism of $X$ :
$$|\mathrm{Fix}(\alpha)| = \sum_{0}^{2\dim X} (-1)^i \tr(\alpha^* | H^i(X)).$$

\subsection{} It is A. Weil who thought that such methods could yield results about varieties defined over other fields, especially over a finite field $k = \F_q$.
A formal application of the Lefschetz trace formula to iterated the Frobenius map :
$$\nu_n = \sum_0^{2\dim X} (-1)^i \tr((\Fr_X^n)^* | H^i(X))$$
would lead to an expression of the zeta function of $X$ as a rational function :
$$Z_X(t) = \frac{\det(1 - t \Fr_X^* | H^-(X))}{\det(1 - t \Fr_X^* | H^+(X))}.$$
Proving such a formula was one of the main goals of the construction of étale cohomology.
This goal was attained in 1963 by Grothendieck and M. Artin.

In fact, there is an étale cohomology with coefficients in each field $\Q_{\ell}$ of $\ell$-adic numbers for $\ell \neq p$ where $p$ is the characteristic of $k$.
This plethora makes the situation more obscure: in addition to the awkwardly arbitrary choice of $\ell$, we don't really know how to relate these $\ell$-adic cohomologies.

Ideally, we would want a \emph{universal cohomology theory} with rational coefficients satisfying the Künneth formula.
However, such a theory with values in $\Q$-vector spaces cannot exist if $p > 0$, because of functoriality considerations about supersingular elliptic curves\footnote{this is well-known work by J.-P. Serre, see below \ref{MISSING REF 6.2}.}.
There is still hope to find such a \enquote{generalized} cohomology theory with values in a suitable semi-simple $\Q$-linear monoidal category in lieu of $\Q$-vector spaces.

\subsection{} Grothendieck suggested to take $\NM^{\eff}(k)_{\Q}$ as such a category\footnote{\enquote{We had the clear impression that in a sense, which remained very vague at first, all these theories should \enquote{be the same}, that they \enquote{gave the same results}. It is to express this intuitive idea of \enquote{kinship} between different cohomology theories that I introduced the notion of \enquote{motive} associated to an algebraic variety. By this term I intend to suggest that it is the \enquote{common motive} (or the common reason) underlying this multitude of cohomological invariants associated to the variety [...] Thus, the motive associated to an algebraic variety would constitute the \enquote{ultimate} cohomological invariant, from which all the others (associated to the different possible cohomology theories) would be deduced, as different \enquote{musical incarnations}, or different \enquote{realizations}...} \cite[A. Grothendieck, Récoltes et semailles, § 16]{recoltes}}.
There is an obvious contravariant monoidal functor
$$\h : \{\text{smooth projective }k\text{-varieties}\} \to \NM^{\eff}(k)_{\Q}$$
mapping a variety to itself ($\h(X)$ is called the motive of $X$), and mapping a morphism to the transpose of its graph modulo numerical equivalence.
We expect that $\h$ provides a universal cohomology theory for smooth projective varieties, and in particular that each $\ell$-adic étale cohomology as well as each reasonable \enquote{Weil} cohomology like crystalline cohomology would factor through $\h$ :
% https://q.uiver.app/#q=WzAsNCxbMCwwLCJYIl0sWzIsMCwiXFxoKFgpIl0sWzIsMiwiSChYKSA9IEhfe1xcZXR9KFggXFxvdGltZXMgXFxvdmVybGluZXtrfSwgXFxRX3tcXGVsbH0pIl0sWzMsMSwiKFxcZWxsXFx0ZXh0ey1hZGljIHJlYWxpemF0aW9uKX0iXSxbMCwxXSxbMCwyXSxbMSwyLCIiLDAseyJzdHlsZSI6eyJib2R5Ijp7Im5hbWUiOiJkYXNoZWQifX19XV0=
\[\begin{tikzcd}
	X && {\h(X)} \\
	&&& {(\ell\text{-adic realization)}} \\
	&& {H(X) = H_{\et}(X \otimes \overline{k}, \Q_{\ell})}
	\arrow[from=1-1, to=1-3]
	\arrow[from=1-1, to=3-3]
	\arrow[dashed, from=1-3, to=3-3]
\end{tikzcd}\]
This expected factorization property means exactly that for each reasonable\footnote{\enquote{reasonable} might even be over-cautiousness.} \enquote{Weil cohomology}, \emph{the homological and the numerical equivalences agree} :
$$\sim_{\hom} \iff \sim_{\num}\ ?$$
It is a deep conjecture in algebraic cycles theory (one of Grothendieck's standard conjectures, see \ref{MISSING REF 2.4} below).

Accepting this conjecture, one can show that the category $\NM^{\eff}(k)_{\Q}$ is graded in a natural way, and one can give a geometric meaning to the decomposition $H(X) = \bigoplus H^i(X)$ and also to the factorization of the zeta function when $k = \F_q$ :
$$Z_X(t) = \frac{\prod_{i\text{ odd}} \det(1 - t\Fr_X^* | H^i(X))}{\prod_{i\text{ even}} \det(1 - t\Fr_X^* | H^i(X))}.$$
Namely, it is a reflection of a decomposition $\h(X) = \bigoplus \h^i(X)$ in $\NM^{\eff}(k)_{\Q}$.

\paragraph*{Basic examples}
\begin{enumerate}
    \item The elementary formula
    $$|\P^n(\F_q)| = 1 + q + \dots + q^n$$
    expresses the fact that the motive of $\P^n$ decomposes as
    $$\h(\P^n) = \1 \oplus \1(-1) \oplus \dots \oplus \1(-n),$$
    where $\1 = \h(\Spec k)$ and $\1(-r) = (\h^2(\P^1))^{\otimes r}$.

    \item The motive of a smooth projective curve $X$ decomposes as
    $$\h(X) = \1 \oplus \h^1(X) \oplus \1(-1).$$
    The $\h^1(X)$ component is a substitute for the jacobian $J(X)$ of $X$ up to isogeny, in the sense where for any two curves $X$ and $X'$,
    $$\NM^{\eff}(k)_{\Q}(\h^1(X), \h^1(X')) \cong \Hom(J(X), J'(X)) \otimes \Q.$$
    Understanding and generalizing Weil's idea of using the jacobian as a geometric substitute for the cohomology $H^1(X, \Q)$ has been incidentally one of the first clues leading Grothendieck to the notion of motive.
\end{enumerate}

The rôle of motives in the \emph{factorization of zeta functions} of varieties over finite fields or number fields is a captivating aspect of the theory, reminiscent of Artin's formalism on factorizations of Dedekind's zeta functions in $L$-functions.
In fact, the \enquote{arithmetical} part of the theory of motives provides a vast generalization (partly conjectural) in higher dimension of Artin's theory.

\section{Galois theory}

\subsection{} Let $k$ be a field, $\overline{k}$ a separable closure, and $\Gal(\overline{k}/k)$ the absolute Galois group of $k$.
Let $k'$ be a finite-dimensional commutative $k$-algebra.
Recall that the finite $k$-scheme $Z = \Spec k'$ is said to be \emph{étale} if the following equivalent properties are satisfied :
\begin{enumerate}[label=(\roman*)]
    \item $k' \otimes \overline{k} \cong \overline{k}^{[k' : k]}$,
    \item $k' \cong \prod k_{\alpha}$ where $k_{\alpha}/k$ are finite separable extensions,
    \item $|Z(\overline{k})| = [k' : k]$.
\end{enumerate}
If moreover $k$ is perfect, these properties are also equivalent to
\begin{enumerate}[resume, label=(\roman*)]
    \item $k'$ is reduced.
\end{enumerate}

The functor
$$\{\text{finite étale }k\text{-schemes}\} \to \{\text{finite continuous }\Gal(\overline{k}/k)\text{-sets}\}$$
given by $Z \mapsto Z(\overline{k})$ is an equivalence of categories known as the \enquote{Galois-Grothendieck correspondence}.

\subsection{} It happens more often that we have to deal with linear representations of $\Gal(\overline{k}/k)$ than with finite $\Gal(\overline{k}/k)$-sets.
As an example, the finite extensions of $k$ already provide representations of $\Gal(\overline{k}/k)$, but there are also Frobenius characters, Artin $L$-series, ...

This suggests to \enquote{linearize} the Galois-Grothendieck correspondence, that is to replace it with an equivalence of the form
$$
    \{\ ?\ \}
    \xrightarrow{\sim}
    \left\{\begin{aligned}\text{finite-dimensional }\Q\text{-vector spaces with}\\
    \text{a continuous linear action of }\Gal(\overline{k}/k)\end{aligned}\right\}
$$
the mysterious category $\{\ ?\ \}$ being semi-simple monoidal $\Q$-linear and linked to finite étale $k$-schemes.
As the latter are simply the $0$-dimensional smooth projective varieties, one idea would be to take the subcategory of $\NM^{\eff}(k)_{\Q}$ spanned by the $0$-dimensional varieties. These objects are called \emph{Artin motives}.

It is not difficult to check that the category $\AM(k)_{\Q}$ of Artin motives provides an answer to this linearization problem. More precisely, there is a commutative diagram
% https://q.uiver.app/#q=WzAsNCxbMCwwLCJcXHtcXHRleHR7ZmluaXRlIMOpdGFsZSB9a1xcdGV4dHstc2NoZW1lc31cXH0iXSxbMiwwLCJcXHtcXHRleHR7ZmluaXRlIGNvbnRpbnVvdXMgfVxcR2FsKFxcb3ZlcmxpbmV7a30vaylcXHRleHR7LXNldHN9XFx9Il0sWzAsMiwiXFxBTShrKV97XFxRfSJdLFsyLDIsIlxce1xcdGV4dHtmaW5pdGUtZGltZW5zaW9uYWwgfVxcUVxcdGV4dHstdmVjdG9yIHNwYWNlc31cXFxcXFx0ZXh0e3dpdGggYSBjb250aW51b3VzIGxpbmVhciBhY3Rpb24gb2YgfVxcR2FsKFxcb3ZlcmxpbmV7a30vaylcXH0iXSxbMCwxLCJcXHNpbSJdLFsyLDMsIlxcc2ltIl0sWzAsMiwiXFxoIiwyXSxbMSwzLCJcXG1hdGhmcmFre2x9Il1d
\[\begin{tikzcd}
	{\{\text{finite étale }k\text{-schemes}\}} && {\{\text{finite continuous }\Gal(\overline{k}/k)\text{-sets}\}} \\
	\\
	{\AM(k)_{\Q}} && \left\{\begin{aligned}\text{finite-dimensional }\Q\text{-vector spaces with}\\
        \text{a continuous linear action of }\Gal(\overline{k}/k)\end{aligned}\right\}
	\arrow["\sim", from=1-1, to=1-3]
	\arrow["\h"', from=1-1, to=3-1]
	\arrow["{\mathfrak{l}}", from=1-3, to=3-3]
	\arrow["\sim", from=3-1, to=3-3]
\end{tikzcd}\]
where $\mathfrak{l}$ is the contravariant linearization functor given by $S \mapsto \Q^S$, an element $g \in \Gal(\overline{k}/k)$ acting on $f \in \Q^S$ by $(g(f))(s) = f(g^{-1}(s))$.

Moreover, the tensor product in $\AM(k)_{\Q}$ corresponding to the cartesian product of $k$-varieties of dimension $0$ is compatible with the tensor product of representations of $\Gal(\overline{k}/k)$.

\subsection{} Changing the point of view, all of this suggests to replace $\AM(k)_{\Q}$ by the more general $\NM^{\eff}(k)_{\Q}$ and to gain a \emph{higher dimensional Galois theory}, namely a Galois theory for systems of multivariate polynomials.
It is the root of Grothendieck's idea of \emph{motivic Galois groups}, which will be outlined in chapter \ref{MISSING REF 6}. In this theory, the (pro)finite groups are replaced by (pro)algebraic groups.

Just as usual Galois theory translates problems about étale algebras into questions about finite or profinite groups, motivic Galois theory aims to reduce a whole class of geometric problems down to questions in the classical theory of representations of reductive groups and their Lie algebras.
This philosophy has inspired, either directly or indirectly, numerous research projects in the last thirty years.
\end{document}