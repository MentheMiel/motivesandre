\documentclass[../main.tex]{subfiles}

\begin{document}

In this preliminary chapter, we present an overview of the theory of rigid $\otimes$-categories and of Tannakian categories, underlying the Galois theory of motives.

Tannakian theory is a powerful and versatile tool which plays for motives the rôle that usual Galois theory plays in arithmetic, or that fundamental groups play in topology.
The theory is itself inspired by these two situations.

\section{Introduction}

Start from ordinary Galois theory: above (\ref{MISSING REF 1.3}) we have considered the equivalent categories
$$\{\text{finite étale }k\text{-schemes}\} \cong \{\text{finite continuous }\Gal(\overline{k}/k)\text{-sets}\}.$$
How to reconstruct the group $\Gal(\overline{k}/k)$ from these categories ?
The answer is well-known and leads to Grothendieck's point of view on fundamental groups in algebraic geometry: $\Gal(\overline{k}/k)$ is the automorphism group of the forgetful functor
$$\{\text{finite continuous }\Gal(\overline{k}/k)\text{-sets}\} \to \{\text{finite sets}\}.$$

After $\Q$-linearization, we obtained the following equivalent categories
$$
    \AM(k)_{\Q}
    \xrightarrow{\sim}
    \left\{\begin{aligned}\text{finite-dimensional }\Q\text{-vector spaces with}\\
    \text{a continuous linear action of }\Gal(\overline{k}/k)\end{aligned}\right\}
$$
where the category on the left is that of Artin motives over $k$, and the same problem of reconstructing $\Gal(\overline{k}/k)$ from these categories arises.
In this context, classical Tannaka-Krein duality for locally compact groups suggests to take into account the tensor product, namely the monoidal structure on these categories.
Indeed, we recover the group $\Gal(\overline{k}/k)$ as the automorphism group of the forgetful $\otimes$-functor
$$
    \left\{\begin{aligned}\text{finite-dimensional }\Q\text{-vector spaces with}\\
    \text{a continuous linear action of }\Gal(\overline{k}/k)\end{aligned}\right\}
    \to
    \{\text{finite-dimensional }\Q\text{-vector spaces}\}.
$$

Suppose for now that $k \subset \C$.
In this case, the space $\Q^{Z(\overline{k})}$ attached to each finite étale $k$-scheme (or more generally, to each Artin motive) is identified to its Betti cohomology $H_B(Z) = H^0(Z(\C), \Q)$ in such a way that $\Gal(\overline{k}/k) \cong \Aut^{\otimes} H_{B|\AM(k)_{\Q}}$.
This suggests a definition of the absolute pure motivic Galois group $G_{\mot, k}$ in higher dimension by replacing $\AM(k)_{\Q}$ with the $\otimes$-category $\NM^{\eff}(k)_{\Q}$ of effective numerical motives (see \ref{MISSING REF 1.1.3}).
Setting up a correct definition of $G_{\mot, k}$ is the origin of Tannakian category theory, initiated by Grothendieck (and developed by N. Saavedra \cite{saavedra72} and P. Deligne \cite{deligne90}).

\section{Rigid $\otimes$-categories}

\subsection{} In this theory the primordial object of study is the category $Rep_F G$ of finite-dimensional representations over a field $F$ of an affine $F$-group scheme $G$.
The algebra of functions on $G$ inherits a coalgebra structure, making it a commutative Hopf algebra $A_G$, so that $G = \Spec A_G$.

The starting point is that the $F$-linear abelian category $Rep_F G$ is equivalent to the category of comodules of finite $F$-dimension over the \emph{coalgebra} $A_G$. The \emph{algebra} structure on $A_G$ is hidden in the monoidal structure $\otimes : Rep_F G \times Rep_F G \to Rep_F G$.

\subsection{} We call \emph{$\otimes$-category}\footnote{unitary symmetric monoidal $F$-linear category in \cite{saavedra72}.} over a commutative ring $F$ any $F$-linear category $\mathcal{T}$ endowed with a $\otimes$-structure, that is :
\begin{enumerate}[label=\roman*)]
    \item a bilinear bifunctor $\otimes : \mathcal{T} \times \mathcal{T} \to \mathcal{T}$,
    \item a unit object $\1$,
    \item natural isomorphisms
    \begin{align*}
        a_{LMN} &: L \otimes (M \otimes N) \xrightarrow{\sim} (L \otimes M) \otimes N\\
        c_{MN} &: M \otimes N \xrightarrow{\sim} N \otimes M\quad\text{such that }c_{NM} = c_{MN}^{-1}\\
        u_M &: M \otimes \1 \xrightarrow{\sim} M,\quad u'_M : \1 \otimes M \xrightarrow{\sim} M
    \end{align*}
    with coherence axioms described by several evident commutative diagrams\footnote{a triangle with vertices $(M \otimes \1) \otimes N$, $M \otimes (1 \otimes N)$, $M \otimes N$, a pentagon whose vertices are different ways of parenthesizing $K \otimes L \otimes M \otimes N$, and a hexagon connecting the cyclic permutations of $L \otimes M \otimes N$, see \cite{saavedra72}}.
\end{enumerate}

A \emph{rigid}\footnote{also known as \emph{autonomous}.} $\otimes$-category over $F$ is a $\otimes$-category $\mathcal{T}$ over $F$ endowed with
\begin{enumerate}[resume, label=\roman*)]
    \item an autoduality $^{\vee} : \mathcal{T} \to \mathcal{T}^{\op}$ such that for each object $M$, the functor $? \otimes M^{\vee}$ is left adjoint to $? \otimes M$, and $M^{\vee} \otimes ?$ is right adjoint to $M \otimes ?$.
\end{enumerate}

Notice that this fourth datum provides natural transformations $\varepsilon : M \otimes M^{\vee} \to \1$ and $\eta : \1 \to M^{\vee} \otimes M$ called respectively evaluation and coevaluation.
They satisfy the relations $\varepsilon \otimes \id_M \circ \id_M \otimes \eta = \id_M$ and $\id_{M^{\vee}} \otimes \varepsilon \circ \eta \otimes \id_{M^{\vee}} = \id_{M^{\vee}}$.
For a morphism $f$, we also write $^tf$ instead of $f^{\vee}$.

In a rigid $\otimes$-category, each endomorphism $f$ has a \emph{trace} defined as the following element of the commutative $F$-algebra $\End(\1)$, obtained as the composition
$$\1 \xrightarrow{f} M^{\vee} \otimes M \xrightarrow{c_{M^{\vee}M}} M \otimes M^{\vee} \xrightarrow{\varepsilon} \1.$$
The \emph{rank} or \emph{dimension} of the object $M$ is the trace of the identity $\id_M$, that is the element $\varepsilon \circ c_{M^{\vee}M} \circ \eta \in \End(\1)$.

\subsubsection{Examples.}
\begin{enumerate}[label=\arabic*)]
    \item For a field $F$ and $\mathcal{T} = Rep_F G$, the unit $\1$ is the trivial representation $F$ and the notion of rank agrees with the usual one.
    \item Let $VecGr_K$ (resp. $sVec_K$) the category of finite-dimensional $\Z$-graded (resp. $\Z/2$-graded) vector spaces over a field $K$.
    The tensor product $\otimes_K$ makes it a rigid $\otimes$-category, the commutativity constraint $c_{MN}$ swapping factors with the Koszul sign rule\footnote{in other words, there is a minus sign when swapping two odd factors.}. The unit is $K$ concentrated in degree $0$. The rank of an object $V^*$ is its super-dimension $\dim V^+ - \dim V^-$, where $V^+ = \bigoplus V^{2i}$ and $V^- = \bigoplus V^{2i+1}$.
    \item Combining the previous two examples, we obtain the rigid $\otimes$-category of super-representations of $G$. In practice, we rather consider the sub-rigid $\otimes$-category $Rep_F(G, \varepsilon)$ defined in the following way. Since $\varepsilon$ is in the center of $G(F)$ and of order $1$ or $2$, $Rep_F(G, \varepsilon)$ is spanned by the super-representations whose even part is exactly where $\varepsilon$ acts trivially.
    \item Vector bundles over a given $F$-scheme $X$ form a rigid $\otimes$-category with $\End(\1) = F$ when $X$ is proper and geometrically connected. The rank agrees with the usual notion.
\end{enumerate}

\subsubsection{Remark.} In a pseudo-abelian rigid $\otimes$-category over a field $F$ of characteristic zero, we can define symmetric and exterior powers of each object $M$ :
$$S^n M = s_n(M),\quad \bigwedge^n M = \lambda_n(M),$$
where $s_n = \frac{1}{n!} \sum_{\sigma \in \Sigma_n} \sigma$ (resp. $\lambda_n = \frac{1}{n!} \sum_{\sigma \in \Sigma_n} \sgn(\sigma)\sigma$) denotes the symmetrization (resp. antisymmetrization) projection.

\subsection{} A \emph{$\otimes$-functor} between $\otimes$-categories is the data of an $F$-linear functor
$$\omega : \mathcal{T} \to \mathcal{T}'$$
and of a collection of natural isomorphisms
\begin{align*}
    o_{MN} : \omega(M \otimes N) &\xrightarrow{\sim} \omega(M) \otimes \omega(N)\\
    o_{\1} : \omega(\1) &\xrightarrow{\sim} \1'
\end{align*}
compatible with the constraints $a$, $c$, $u$ and $u'$.

If the $\otimes$-categories are rigid, then such a functor is automatically compatible with the duality $^{\vee}$ (up to natural isomorphism). Moreover, thanks to this duality, every natural transformation between $\otimes$-functors is then a natural isomorphism (see \cite{deligne90}).

\subsubsection{Exercise.} Construct the free rigid $\otimes$-category generated by one object $M$.

\section{Tannakian categories}

\subsection{} Let $F$ be a field and $\mathcal{T}$ be an abelian rigid $\otimes$-category over $F$ such that $\End(\1) = F$.
A \emph{fiber functor} on $\mathcal{T}$ is a faithful exact $\otimes$-functor
$$\omega : \mathcal{T} \to Vec_K$$
whose target is the rigid $\otimes$-category of finite-dimensional vector spaces over a certain field extension $K$ of $F$.
If such a functor exists, then we say that $\mathcal{T}$ is \emph{Tannakian}\footnote{in reference to classical Tannaka-Krein theory for compact groups. It deserves as well, if not more, to be called a Kreinian category, but usage decided otherwise, see \cite{bre94}. The Tannakian part of the theory provides a left inverse to $G \mapsto Rep_F G$ and the Kreinian part tells us that it is also a right inverse.}.

We can then define the affine $K$-group scheme $G = \uAut^{\otimes} \omega$ called the \emph{Tannakian group} of $\mathcal{T}$ associated with $\omega$.
For each extension $K'/K$, $(\uAut^{\otimes} \omega)(K')$ is the automorphism group of the extended $\otimes$-functor $\omega_{K'} : \mathcal{T} \to Vec_{K'}$.

\subsubsection{Theorem (Deligne \cite{deligne90}, 7).} Let $\mathcal{T}$ be a rigid $\otimes$-category over a field $F$ of characteristic zero, which is abelian and such that $\End(\1) = F$.
Then $\mathcal{T}$ is Tannakian if and only if the rank of every object is a natural number, if and only if for each object $M$, we have $\bigwedge^n M = 0$ for $n$ large enough.

\subsection{} If there exists a fiber functor $\omega$ with $K = F$, we say that $\mathcal{T}$ is \emph{neutral Tannakian}.

In this case, $\omega$ becomes an \emph{equivalence} of rigid $\otimes$-categories
$$\omega : \mathcal{T} \to Rep_F G$$
where $Rep_F G$ denotes the rigid $\otimes$-category of finite-dimensional $F$-representations of $G = \uAut^{\otimes} \omega$.
Moreover, the category of fiber functors $\omega$ over $F$ is equivalent to the groupoid of $G$-torsors.

This establishes a correspondence between $\otimes$-categorical properties and properties of the associated groups.
Here are some examples :
\begin{itemize}
    \item $G$ is an algebraic group (that is, $G$ is of finite type over $F$) if and only if $\mathcal{T}$ admits a $\otimes$-generator.
    This means that there is an object $M$ such that every object of $\mathcal{T}$ is a subquotient of a finite sum $\bigoplus (M^{\otimes m_i} \otimes M^{\vee, \otimes n_i})$.
    In this case, we write $\mathcal{T} = \langle M \rangle^{\otimes}$ and $G$ is identified with a Zariski closed algebraic subgroup of $GL(\omega(M))$.
    A subspace of $\bigoplus (\omega(M)^{\otimes m_i} \otimes (\omega(M)^{\vee})^{\otimes n_i})$ is stable under the action of $G$ if and only if it is the image by $\omega$ of a subobject of $\bigoplus (M^{\otimes m_i} \otimes M^{\vee, \otimes n_i})$.
    \item (if the characteristic of $F$ is zero) $G$ is a pro-reductive\footnote{An affine scheme group $G$ over a field $F$ of characteristic zero is pro-reductive if its unipotent radical is trivial. This is equivalent to $G$ being a inverse limit of reductive algebraic $F$-groups (non necessarily connected).} if and only if $\mathcal{T}$ is semi-simple.
\end{itemize}

\subsection{} Let $\phi : \mathcal{T}' \to \mathcal{T}$ be an exact $\otimes$-functor between neutral Tannakian categories.
Let $\omega : \mathcal{T} \to Vec_F$ be a fiber functor.
Then $\omega' = \omega \circ \phi : \mathcal{T}' \to Vec_F$ is also a fiber functor, and there is a homomorphism
$$f = \phi^* : G = \uAut^{\otimes} \omega \to G' = \uAut^{\otimes} \omega'$$
between the associated Tannakian groups.

Conversely, each group homomorphism $f : G \to G'$ yields a $\otimes$-functor $\phi = f^* : Rep_F G' \to Rep_F G$.
We have that :
\begin{itemize}
    \item $f$ is a monomorphism (a closed immersion) if and only if each object $M$ of $\mathcal{T}$ is a subquotient of the image by $\phi$ of an object $N'$ of $\mathcal{T}'$,
    \item $f$ is an epimorphism (faithfully flat) if and only if $\phi$ is fully faithful and for each object $M'$ of $\mathcal{T}'$, each subobject of $\phi(M')$ is the image by $\phi$ of a subobject of $M'$ (this last condition is automatically satisfied if $\mathcal{T}$ is semi-simple).
    \item $\phi$ identifies $\mathcal{T}'$ with the category of objects of $\mathcal{T}$ endowed with an action of $G'$ factoring the action of $G$.
\end{itemize}

\subsection{} The Tannakian group $G$ is not uniquely determined by $\mathcal{T}$ (for example, two $F$-groups which are inner forms of each other have equivalent $\otimes$-categories of representations), the datum $\omega$ is essential.

In a more intrisic and general manner, we can attach to each Tannakian category\footnote{or even more generally, any abelian rigid $\otimes$-category such that $\End(\1) = F$ perfect, whose objects and morphism $F$-spaces are of finite length.} $\mathcal{T}$ over $F$ a commutative Hopf algebra $\mathcal{O}(\pi(\mathcal{T}))$ in the category of ind-objects of $\mathcal{T}$, such that for every fiber functor $\omega : \mathcal{T} \to Vec_K$, $\omega(\mathcal{O}(\pi(\mathcal{T})))$ is the algebra of functions of the $K$-group scheme $\uAut^{\otimes} \omega$.

Denote by $\pi(\mathcal{T})$ the same object in the opposite category (that of pro-objects of $\mathcal{T}^{\op}$) seen as an \enquote{affine $\mathcal{T}$-group scheme}.
As such, $\omega(\pi(\mathcal{T}))$ is identified with the usual affine $K$-group scheme $\uAut^{\otimes} \omega$.
If $\mathcal{O}(\pi(\mathcal{T}))$ is cocommutative, then $\pi(\mathcal{T})$ itself can be seen as an affine $K$-group scheme in the usual sense.

Each exact $\otimes$-functor $\phi : \mathcal{T}' \to \mathcal{T}$ between Tannakian categories over $F$ yields a homomorphism $\pi(\mathcal{T}) \to \phi\pi(\mathcal{T}')$ and identifies $\mathcal{T}'$ with the category of objects of $\mathcal{T}$ endowed with an action of $\phi\pi(\mathcal{T}')$ factoring the action of $\pi(\mathcal{T})$, see \cite{deligne90}.

\subsection{} A \emph{Tannakian subcategory} of a Tannakian category $\mathcal{T}$ is a full subcategory $\mathcal{T}'$ of $\mathcal{T}$ stable by $\otimes$ and $^{\vee}$ such that each subobject and quotient in $\mathcal{T}$ of an object of $\mathcal{T}'$ lies in $\mathcal{T}'$\footnote{as a counter-example, the fully faithful restriction $\otimes$-functor $Rep_F GL_n \to Rep_F B_n$, where $B_n$ is the subgroup of triangular matrices, does not make $Rep_F GL_n$ a Tannakian subcategory.}.
The corresponding homomorphim $\mathcal{O}(\pi(\mathcal{T}')) \to \mathcal{O}(\pi(\mathcal{T}))$ is then faithfully flat, and for every fiber functor on $\mathcal{T}$ we get a faithfully flat homomorphism between the Tannakian groups.
\end{document}