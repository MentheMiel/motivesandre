\documentclass[../main.tex]{subfiles}

\begin{document}

In this second preliminary chapter, we present the basic tools of the theory of pure motives: algebraic cycles, adequate equivalences and Weil cohomologies.

\section{Algebraic cycles and adequate relations}

\subsection{} Let $k$ be a base field and denote $\mathcal{P}(k)$ the \emph{category of smooth projective schemes over $k$}, also sometimes called smooth projective $k$-varieties from now on.

For each $X \in \mathcal{P}(k)$ let $\mathcal{Z}^*(X)$ be the graded group of algebraic cycles on $X$, that is, the free abelian group generated by the integral closed subschemes $Z$ of $X$ and graded by codimension\footnote{we do not ask of $X$ that it is connected or even equidimensional, the codimension of an integral subscheme is still well-defined.}.
Denote by $[Z]$ the image of the subscheme $Z$ in $\mathcal{Z}^*(X)$ (and in its quotients).

For each commutative ring $F$, the elements of
$$\mathcal{Z}^r(X)_F = \mathcal{Z}^r(X) \otimes_{\Z} F$$
are called the \emph{algebraic cycles of codimension\footnote{we avoid the word \enquote{degree} to prevent any confusion with the degree of $0$-cycles defined in \ref{MISSING REF 3.2.6} or with the degree of the cycle class in cohomology, which is twice the codimension. Furthermore, it is also useful to consider the grading defined by dimension instead of codimension.} $r$ with coefficients in $F$}.

Intersection theory constructs a partially defined bilinear composition law on algebraic cycles: the intersection product, counting multiplicites, of two cycles whose components intersect properly (see \cite{se57}).
To extend this definition to all cycles, the classical approach consists in using an adequate equivalence relation, see \cite{sa58}.

\subsubsection{Definition.} An equivalence relation $\sim$ on algebraic cycles is said to be \emph{adequate} if it satisfies the following conditions for each $X, Y \in \mathcal{P}(k)$ :
\begin{enumerate}[label=\arabic*)]
    \item $\sim$ is compatible with the $F$-linear structure and with the grading,
    \item for each $\alpha, \beta \in \mathcal{Z}^*(X)_F$ there exists an $\alpha' \sim \alpha$ such that $\alpha'$ and $\beta$ intersect properly (so that the intersection product $\alpha' \cdot \beta$ is well-defined),
    \item for each $\alpha \in \mathcal{Z}^*(X)_F$ and each $\gamma \in \mathcal{Z}^*(X \times Y)_F$ intersecting properly $(\pr^{XY}_X)^{-1}(\alpha)$, we have $\alpha = 0 \implies \gamma_*(\alpha) \sim 0$ where
    $$\gamma_*(\alpha) = \pr^{XY}_Y(\gamma \cdot (\pr^{XY}_X)^{-1}(\alpha)).$$
\end{enumerate}

\subsubsection{Exercise.} Show that these conditions ensure that an \enquote{external} product $\alpha \times \beta$ (on $X \times Y$) is $\sim 0$ as soon as $\alpha \sim 0$ or $\beta \sim 0$.

\subsection{} These conditions ensure that the intersection product which is partially defined on $\mathcal{Z}^*(X)_F$ factors through the quotient by $\sim$, and yields a well-defined composition law on
$$\mathcal{Z}^*_{\sim}(X)_F = \mathcal{Z}^*(X)_F/\sim$$
endowing it with the structure of a \emph{commutative graded $F$-algebra} (and not graded-commutative)\footnote{other notations often used are $A^*_{\sim}(X)_F$, $C^*_{\sim}(X)_F$, $A^*_{\sim}(X, F)$, ...}.

\emph{N.B.} One shall be careful not to confuse $\mathcal{Z}^*_{\sim}(X) \otimes_{\Z} F$ with its quotient $\mathcal{Z}^*_{\sim}(X)_F$.

The third condition defining adequate relations implies that constructing the algebra $\mathcal{Z}^*_{\sim}(X)_F$ is \emph{contravariant} in $X$ (let $\gamma$ be the transpose of the graph $\Gamma_f$ of the given morphism $f$, obtained by exchanging in $\Gamma_f$ the factors $X \times Y \to Y \times X$).
We also obtain a homomorphism of $F$-modules \emph{in the other way} $f_* : \mathcal{Z}^*_{\sim}(X)_F \to \mathcal{Z}^*_{\sim}(Y)_F$ shifting the grading by $(- \text{the generic relative dimension of }f)$, if the latter is constant on the components of $X$ (this time, take $\gamma = \Gamma_f$).
This induced homomorphism $f_*$ is not compatible with intersection products, however we have the so-called \emph{projection formula}
$$f_*(\alpha \cdot f^*(\beta)) = f_*(\alpha) \cdot \beta.$$
In addition, for each cartesian square in $\mathcal{P}(k)$
% https://q.uiver.app/#q=WzAsNCxbMCwwLCJYJyJdLFsxLDAsIlknIl0sWzAsMSwiWCJdLFsxLDEsIlkiXSxbMCwxLCJmJyJdLFswLDIsInEiLDJdLFsxLDMsInAiXSxbMiwzLCJmIl1d
\[\begin{tikzcd}
	{X'} & {Y'} \\
	X & Y
	\arrow["{f'}", from=1-1, to=1-2]
	\arrow["q"', from=1-1, to=2-1]
	\arrow["p", from=1-2, to=2-2]
	\arrow["f", from=2-1, to=2-2]
\end{tikzcd}\]
we have the formula $f^*p_* = q_*f'^*$ (see \cite{fu84}).

\subsection{} The elements of
$$\mathcal{Z}^{\dim X + r}(X \times Y)_F,\quad\text{resp.}\quad \mathcal{Z}_{\sim}^{\dim X + r}(X \times Y)_F$$
are called \emph{algebraic correspondences of degree $r$ with coefficients in $F$} (resp. modulo $\sim$) \emph{from $X$ to $Y$} (or between $X$ and $Y$). When $X$ is not equidimensional, one shall consider $\dim X$ as a locally constant function on $X$.

For example, the transpose of the graph of a morphism $Y \to X$ is a correspondence of degree $0$ from $X$ to $Y$.

The formula
$$g \circ f = \pr^{XYZ}_{XZ*} \left(\pr^{XYZ*}_{XY}(f) \cdot \pr^{XYZ*}_{YZ}(g)\right)$$
defines an associative composition law\footnote{that this formula defines $g \circ f$ and not $f \circ g$ is compatible with the above definition of the degree of a correspondence, and is consistent with the traditionally contravariant point of view on motives.} for correspondences modulo $\sim$ (see \cite{fu84})
$$\mathcal{Z}^{\dim Y + r}_{\sim}(X \times Y)_F \otimes_F \mathcal{Z}^{\dim Z + s}_{\sim}(Y \times Z)_F \to \mathcal{Z}^{\dim Z + r + s}_{\sim}(X \times Z)_F,$$
adding the degrees and endowing $\mathcal{Z}^{\dim X}_{\sim}(X \times X)_F$ with the structure of a (non necessarily commutative) $F$-algebra, the algebra of correspondences of degree $0$.
The unit is the class of the diagonal $\Delta_X \subset X \times X$.
This algebra is also endowed with an (anti-)involution given by transposition $^{\top}$.

\subsection{} For $f \in \mathcal{Z}_{\sim}(X \times Y)_F$, $\alpha \in \mathcal{Z}_{\sim}(X)$ and $\beta \in \mathcal{Z}_{\sim}(Y)$, we define generally
\begin{align*}
    f_*(\alpha) &= \pr^{XY}_{Y*}(f \cdot \pr^{XY*}_X (\alpha)) \in \mathcal{Z}_{\sim}(Y),\\
    \text{and } f^*(\beta) &= \pr^{XY}_{X*}(f \cdot \pr^{XY*}_Y(\beta)) \in \mathcal{Z}_{\sim}(X).
\end{align*}
For $\gamma \in \mathcal{Z}^{\dim X + r}_{\sim}(X \times Y)_F$, $a \in \mathcal{Z}^{\dim X' + p}_{\sim}(X \times X')_F$ and $b \in \mathcal{Z}^{\dim Y + q}_{\sim}(Y \times Y')_F$, we have the useful formula:
$$(a, b)^*(\gamma) = b \circ \gamma \circ a^{\top} \in \mathcal{Z}^{\dim X' + p + q + r}_{\sim}(X' \times Y')_F.$$

\subsubsection{Exercise.} Let $k'/k$ be a field extension.
Start with an adequate equivalence relation $\sim$ on algebraic cycles on objects of $\mathcal{P}(k')$.
By \enquote{restriction}, it induces an adequate equivalence relation (still denoted $\sim$) on algebraic cycles $\alpha$ on objects of $\mathcal{P}(k)$: $\alpha \sim 0$ if and only if $\alpha_{k'} \sim 0$.
Whence there is a canonical injective homomorphism
$$\mathcal{Z}^*_{\sim}(X)_F \hookrightarrow \mathcal{Z}^*_{\sim}(X_{k'})_F.$$
When $k' = \overline{k}$, the absolute Galois group $\Gal(\overline{k}/k)$ acts naturally on $\mathcal{Z}^*_{\sim}(X_{\overline{k}})_F$, and the previous homomorphism induces
$$\mathcal{Z}^*_{\sim}(X)_F \hookrightarrow \mathcal{Z}^*_{\sim}(X_{\overline{k}})_F^{\Gal(\overline{k}/k)}.$$
Show that this homomorphism is bijective if $F$ is a $\Q$-algebra, but not surjective in general if $F = \Z$.

\section{Review of the usual adequate relations}

\subsection{} They are the:
\begin{itemize}
    \item rational equivalence $\sim_{\rat}$,
    \item algebraic equivalence $\sim_{\alg}$,
    \item homological equivalences $\sim_{\hom}$,
    \item numerical equivalence $\sim_{\num}$,
\end{itemize}
to which we add the interesting equivalence $\sim_{\otimes\nil}$ of \enquote{smash-nilpotence} (or $\otimes$-nilpotence) appearing in a work of V. Voevodsky \cite{voevodsky95}\footnote{when these symbols appear as subscripts or superscripts, we will shorten $\sim_{\rat}$ as $\rat$, and so on.}.

To compare them, say that $\sim$ is finer than $\approx$, and write $\sim \succ \approx$, if $\alpha \sim 0$ implies $\alpha \approx 0$. We then have
$$\sim_{\rat}\ \succ\ \sim_{\alg}\ \succ\ \sim_{\hom}\ \succ\ \sim_{\num},\text{ and}$$
$$\sim_{\rat}\ \succ\ \sim_{\alg}\ \succ\ \sim_{\otimes\nil}\ \succ\ \sim_{\hom}\ \succ\ \sim_{\num}\text{ if } F \supset \Q.$$
As a foreshadowing, let us mention that we conjecture that the last three equivalences of this sequence coincide exactly (Grothendieck, Voevodsky).

\subsection{} A cycle $\alpha \in \mathcal{Z}^*(X)_F$ is \emph{rationally equivalent to $0$} (that is, $\alpha \sim_{\rat} 0$) if there exists $\beta \in \mathcal{Z}^*(X \times \P^1)_F$ such that $\beta(0)$ and $\beta(\infty)$ are well-defined\footnote{more precisely, such that each component $Z$ of $\beta$ is dominant over $\P^1$, $\beta(0)$ (resp. $\beta(\infty)$) being then the sum of the cycles associated to the closed subschemes $Z(0)$ (resp. $Z(\infty)$) of $X$, see \cite{fu84}.} and such that $\alpha = \beta(0) - \beta(\infty)$.

The fact that $\sim_{\rat}$ is adequate is among the foundations of intersection theory: the difficult condition is the second one, which is W. Chow's \enquote{moving lemma}, see \cite{fu84}.

The graded rings $\mathcal{Z}^*(X)/\sim_{\rat}$ are called the \emph{Chow rings} and are customarily denoted $\CH^*(X)$.
We have $\CH^*(X) \otimes F = \mathcal{Z}^*_{\rat}(X)_F$, also denoted $\CH^*(X)_F$.

\subsubsection{Lemma.} $\sim_{\rat}$ is the finest adequate relation.

Indeed, let $\sim$ be an adequate relation on algebraic cycles with coefficients in $F$.
By the third condition of adequateness, it suffices to show that $[0] \sim [\infty]$ on $\P^1$.
By the second condition, there exists a cycle $\sum n_i[x_i] \sim [1]$ with $n_i \in F$, such that $\sum n_i[x_i] \cdot [1]$ is well-defined (that is, $x_i \neq 1$).
Apply the third condition with $\gamma$ the graph of the polynomial $1 - \prod \left(\frac{x - x_i}{1 - x_i}\right)^{m_i}$ (with $m_i > 0$) and $\alpha = \sum n_i[x_i] - [1]$.
We obtain that $mn[1] \sim m[0]$ where $m = \sum m_i$ and $n = \sum n_i$.
Since the $m_i$ can still be chosen freely, we conclude that $n[1] \sim [0]$.
Now, apply the automorphism $x \mapsto 1/x$ (and the third condition again) to obtain $n[1] \sim [\infty]$, whence $[0] \sim [\infty]$ as required.

In virtue of this lemma, we can identify any adequate relation on algebraic cycles with coefficients in $F$ with the data of a homogeneous ideal $I^*_{\sim}(X) = \{\alpha \in \CH(X)_F, \alpha \sim 0\}$ of the Chow ring $\otimes F$ of every $X \in \mathcal{P}(k)$, satisfying a compatibility condition with the bifunctoriality of Chow groups.

\subsubsection{Exercises.}
\begin{enumerate}[label=\arabic*)]
    \item Let $x \in \P^1(k)$. Show that the diagonal of $\P^1 \times \P^1$ decomposes modulo $\sim$ as the sum of the idempotent correspondences $[x] \times \P^1$ and $\P^1 \times [x]$, and that this decomposition is independent of $x$.
    \item More generally, let $t \in \mathcal{Z}^1_{\sim}(\P^n)_F$ be the class of a hyperplane in $\P^n$, and let $[\Delta] \in \mathcal{Z}^n_{\sim}(\P^n \times \P^n)_F$ be the class of the diagonal.
    Show that $[\Delta] = \sum t^i \times t^{n-i}$.

    Deduce that $\mathcal{Z}^*_{\sim}(X \times \P^n)_F \cong \mathcal{Z}^*_{\sim}(X)_F[t]/(t^{n+1})$ as graded $F$-algebras, $t$ being of degree $1$. Explain how this generalizes Bézout's theorem.
\end{enumerate}

\subsection{} The definition of $\sim_{\alg}$ is analogous to that of $\sim_{\rat}$.
The projective line $\P^1$ is replaced by an arbitrary smooth projective curve (or equivalently\footnote{any two points can be joined by a smooth connected projective curve: cut the variety by hypersurfaces general enough and of high enough degree passing through the two points, so that the section is a smooth connected projective curve (they even exist if $k$ is finite, see \cite{poonen}).} any smooth connected projective $k$-scheme, that is how one can prove that the condition $\sim_{\alg} 0$ is stable under addition) and $0$ and $\infty$ are replaced by two $k$-rational points.
It is easy to see that $\mathcal{Z}^*_{\alg}(X)_F = \mathcal{Z}^*_{\alg}(X) \otimes_{\Z} F$.

\subsection{} According to \cite{voevodsky95}, an element $\alpha \in \mathcal{Z}^*(X)_F$ is \enquote{smash-nilpotent} or $\otimes$-nilpotent if there exists $N > 0$ such that $\alpha \times \alpha \times \dots \times \alpha$ is rationally equivalent to $0$ on $X^N$.
One checks without difficulty that this is indeed an adequate relation\footnote{this is even obvious when using the adequateness criterion given in the next chapter.}, and that $\mathcal{Z}^*_{\otimes\nil}(X)_F = \mathcal{Z}^*_{\otimes\nil}(X) \otimes_{\Z} F$.

\subsubsection{Proposition (\cite{voevodsky95}).} If $F$ is a $\Q$-algebra, then $\sim_{\alg}$ is finer than $\sim_{\otimes\nil}$.
\begin{proof}
Let $\alpha \in \CH^r(X)_F$ be an element sent to $0$ in $\mathcal{Z}^r_{\alg}(X)_F$.
There exists a genus $g$ smooth connected projective curve $T$, two points $t_0$ and $t_1$ and $\beta \in \CH^r(X \times T)_F$, such that $\alpha = \beta(t_0) - \beta(t_1)$.
We have $\alpha^{\otimes N} = \beta^{\otimes N}([t_0] - [t_1])^{\otimes N}$, so that it suffices to prove that $([t_0] - [t_1])^{\otimes N} = 0$ in $\CH^N(T^N)_{\Q}$ for $N$ large enough (in fact, $N = 2g$ suffices if $g \geq 1$, which we can assume).
This relies on the fact that the canonical morphism $S^n(T) \to J(T)$ from the $n$-th symmetric power of $T$ to the Jacobian given by $x_1 + \dots + x_n \mapsto x_1 + \dots + x_n - nt_0$ identifies $S^n(T)$ with a projective bundle on $J(T)$, as soon as $n \geq 2g - 1$ \cite{col75}.
Knowing the structure of the Chow groups of a projective bundle \cite{fu84}, we deduce that the inclusion $\iota : S^{2g - 1}(T) \hookrightarrow S^{2g}(T)$ given by $x_1 + \dots + x_{2g - 1} \mapsto x_1 + \dots + x_{2g - 1} + t_0$ induces an isomorphism $\iota^* : \CH^{2g}(S^{2g}(T)) \cong \CH^{2g-1}(S^{2g-1}(T))$.
Furthermore, $\CH^{2g}(S^{2g}(T))_{\Q}$ is identified with the symmetric elements in $\CH^{2g}(T^{2g})_{\Q}$ \cite{fu84}.
This lets us see $([t_0] - [t_1])^{\otimes 2g}$ as a cycle on $S^{2g}(T)$, and it is easy to see that $\iota^*(([t_0] - [t_1])^{\otimes 2g}) = 0$.
\end{proof}

\subsubsection{Remarks.}
\begin{enumerate}[label=\arabic*)]
    \item The bound $2g$ for the nilpotence exponent is not optimal. One can show that the optimal bound is $g + 1$, see point 2) of \ref{MISSING REF 4.3.3.2}.
    \item Let $A$ be an elliptic curve, and $x_0$ and $x_1$ two distinct points in $A(k)$.
    The above proof shows that the cycle $([x_0] - [x_1]) \times ([x_0] - [x_1])$ on $A \times A$ is zero modulo rational equivalence, while $[x_0] - [x_1]$ on $A$ is not itself zero.
    This disproves the converse of exercise \ref{MISSING REF 3.1.1.2}.
\end{enumerate}

A remarkable aspect of this result is that even though $\sim_{\rat}$ and $\sim_{\otimes\nil}$ seem very similar by definition, the groups of cycles modulo $\sim_{\rat}$ and modulo $\sim_{\otimes\nil}$ have wildly different properties when $k$ \emph{is algebraically closed}.
On the one hand, Chow groups are \enquote{continuous} invariants, in general \enquote{enormous} (and varying with $k$) as shown by D. Mumford.
On the other hand, the groups $\mathcal{Z}^*_{\alg}(X)$ and \emph{a fortiori} $\mathcal{Z}^*_{\sim}(X)$ for every adequate equivalence coarser that $\sim_{\alg}$ (in particular $\sim_{\otimes\nil}$) are \enquote{discrete} invariants, countable and \emph{invariant by extensions of $k$} in virtue of the theory of Chow forms \cite{Kl70a}.

\subsection{} Homological equivalence relies on the notion (and the choice) of a Weil cohomology, which will be explained below. It is the only usual adequate equivalence for which it is unclear \emph{a priori} that $\mathcal{Z}^*_{\sim}(X) \otimes_{\Z} F = \mathcal{Z}^*_{\sim}(X)_F$ for each $F$ contained in the ring of coefficients of the chosen cohomology\footnote{as such, it is also unclear that $\mathcal{Z}^*_{\sim}(X)$ has finite rank, whereas this is obviously the case for $\mathcal{Z}^*_{\sim}(X)_K$ where $K$ is the coefficient ring.}.

\subsection{} We customarily call \emph{$0$-cycle} a cycle of dimension $0$, that is a linear combination $\sum n_i [P_i]$ of closed points. Its degree is defined as
$$\sum n_i [k(P_i) : k].$$
It only depends on the class of the $0$-cycle modulo algebraic equivalence\footnote{and of course, it only depends on its class modulo numerical equivalence, by definition.} (this is a rephrasing of Poncelet's \enquote{conservation of number principle} \cite{fu84}).

\subsection{} An element $\alpha \in \mathcal{Z}^r(X)_F$ is \emph{numerically equivalent to $0$} if for each cycle $\beta$ of dimension $r$, the $0$-cycle $\alpha \cdot \beta$ (well-defined in $\CH^d(X)$) is of degree $\langle \alpha, \beta\rangle$ equal to zero.

It is known that $\mathcal{Z}^r_{\num}(X)_{\Q} = \mathcal{Z}^r_{\alg}(X)_{\Q}$ when $r \leq 1$ (Matsusaka \cite{matsusaka57}, reducing the problem to the case of a surface) but the kernel of the quotient map $\mathcal{Z}^r_{\alg}(X)_{\Q} \to \mathcal{Z}^r_{\num}(X)_{\Q}$ is infinite-dimensional in general if $r \geq 2$ \cite{Cl83}.

\subsubsection{Proposition.} If $F$ is an integral domain of characteristic zero, then the $F$-module $\mathcal{Z}^r_{\num}(X)_F$ is free of finite type and $\mathcal{Z}^r_{\num}(X)_F = \mathcal{Z}^r_{\num}(X) \otimes_{\Z} F$.
What is more, if $F \supset \Q$ and $X$ has pure dimension $d$, then the \enquote{degree of the intersection $0$-cycle} pairing
$$\mathcal{Z}^r_{\num}(X)_F \times \mathcal{Z}^{d-r}_{\num}(X)_F \to F,\quad (\alpha, \beta) \mapsto \langle \alpha, \beta\rangle$$
is a perfect pairing.

Since the proof relies on the existence of Weil cohomologies, we postpone it (to \ref{MISSING REF 3.4.5}).

\subsubsection{Exercise.} Suppose that $F$ is a field. Show that $\sim_{\num}$ is the coarsest non-trivial adequate equivalence relation.

\section{Weil cohomologies}

\subsection{} Taking inspiration from \cite{saavedra72}, we axiomatize the properties of the cohomologies mentioned in \ref{MISSING REF 1.2.2}.

The product $\times_k$ endows the category $\mathcal{P}(k)$ of smooth projective $k$-schemes with the structure of a (non additive) symmetric monoidal category.
The unit is the point $\Spec k$ and there are canonical isomorphisms
$$X \times Y \cong Y \times X,\quad (X \times Y) \times Z \cong X \times (Y \times Z).$$
The diagonal $\Delta_X : X \hookrightarrow X \times X$ endows each object $X \in \mathcal{P}(k)$ with the structure of a counitary cocommutative coalgebra with respect to this monoidal structure.

Recall that $VecGr_K$ denotes the rigid $\otimes$-category of finite-dimensional $\Z$-graded vector spaces over a field $K$ (with commutativity given by the Koszul sign rule).
Let $VecGr_K^{\geq 0}$ be its full (not rigid) sub-$\otimes$-category spanned by objects concentrated in nonnegative degrees.

\subsubsection{Definition.} A (pure) Weil cohomology\footnote{in the literature it is often asked that $K$ is of characteristic zero, however the general case, even that of a product of fields, is useful.} $H^*$ is a functor
$$H^* : \mathcal{P}(k)^{\op} \to VecGr_K^{\geq 0}$$
compatible with the monoidal structure, satisfying
\begin{itemize}[label=\ast)]
    \item $\dim_K H^2(\P^1) = 1$,
\end{itemize}
and endowed with the additional data 1) and 2) given below\footnote{several authors also require $H^*$ to satisfy the \enquote{weak and hard Lefschetz theorems}. We avoid it, on the one hand because we will have to consider generalizations of the notion of Weil cohomologies to the mixed case (\ref{MISSING REF 14.2.4}) where the \enquote{Lefschetz theorems} lose their relevance, and on the other hand because we will construct in the pure case \emph{a priori} non-classical Weil cohomologies for which the \enquote{Lefschetz theorems} are not yet known to hold (\ref{MISSING REF 9.1.4}).}.

Notice that the comultiplication of each object $X$ of $\mathcal{P}(k)$ induces \emph{ipso facto} a multiplication (the cup-product) on $H^*(X)$, making $H^*(X)$ a (unitary) graded-commutative $\mathbf{N}$-graded $K$-algebra (that is, a unitary commutative algebra object in $VecGr_K^{\geq 0}$).

For each $d \in \Z$, denote $(r)$ the \enquote{Tate twist} operation $V^* \mapsto V^* \otimes H^2(\P^1)^{\otimes (-r)}$ in $VecGr_K$ (considering that the $\ast$) condition implies the $\otimes$-invertibility of $H^2(\P^1)$ in $VecGr_K$).

The additional data in \ref{MISSING REF 3.3.1.1} are:
\begin{enumerate}[label=\arabic*)]
    \item (trace, Poincaré duality) for each $X \in \mathcal{P}(k)$ with pure dimension $d$, a $K$-linear map $H^{2d}(X)(d) \xrightarrow{\Tr_X} K$, which is an isomorphism when $X$ is geometrically connected, satisfying $\Tr_{X \times Y} = \Tr_X \Tr_Y$ (after the obvious identifications that are needed) and such that the product of $H^*(X)$ induces for each $i$ a duality pairing
    $$\langle\ ,\ \rangle : H^i(X) \times H^{2d - i}(X)(d) \to H^{2d}(X)(d) \xrightarrow{\Tr_X} K.$$
    \item (cycle classes) for each $X \in \mathcal{P}(k)$, abelian group homomorphisms
    $$\gamma_X^r = \gamma_{X, H}^r : \CH^r(X) \to H^{2r}(X)(r)$$
    that are
    \begin{itemize}
        \item contravariant in $X$,
        \item compatible with the \enquote{external product} $\gamma_{X \times Y}^{r + s}(\alpha \times \beta) = \gamma_X^r(\alpha) \otimes \gamma_Y^s(\beta)$,
        \item \enquote{normalized} such that $\gamma^d$ composed with the trace $\Tr_X$ coincides with the $0$-cycles degree map considered in \ref{MISSING REF 3.2.6} when $X$ has pure dimension $d$.
    \end{itemize}
\end{enumerate}

Compatibility with the monoidal structure means that for each pair $(X, Y)$ there exists a canonical Künneth isomorphism between graded-commutative graded algebras:
$$H^*(X \times Y) \cong H^*(X) \otimes H^*(Y)$$
functorially in $X$ and $Y$ (and compatible with the associativity and commutativity constraints).
The projections $\pi_X^i = \pi_{X, H}^i : H^*(X) \to H^i(X) \hookrightarrow H^*(X)$ on homogeneous components are called the \emph{Künneth projectors} of $X$.

\subsection{Some formal consequences of the definition}
See \cite{Kl68}, \cite{Kl94}.
\begin{itemize}
    \item For each $r \in \Z$, $K(r)$ is isomorphic (not canonically in general) to $K$ concentrated in degree $-2r$. The double Tate torsion $(r)(s)$ is the same as the simple torsion $(r + s)$.
    \item $H^*(\Spec k) = H^0(\Spec k) \stackrel{\Tr_{\Spec k}}{=} K$.
    \item $H^i(X) = 0$ for $i > 2 \dim X$.
    \item Poincaré duality can also be written in the following form, for each $r \in \Z$
    $$\langle\ ,\ \rangle : H^i(X)(r) \times H^{2d_X - i}(X)(d - r) \to H^{2d_X}(X)(d) \xrightarrow{\Tr_X} K,$$
    and we have $\langle x, x'\rangle = (-1)^i \langle x', x\rangle$ for each $x \in H^i(X)(r)$ and $x' \in H^{2d-i}(X)(d-r)$.
    \item $\langle \gamma_r(\alpha), \gamma_{d - r}(\beta)\rangle = \langle \alpha, \beta\rangle$.
    \item Let $X, Y \in \mathcal{P}(k)$ with dimensions $d_X$ and $d_Y$ respectively, and $f : X \to Y$ be a morphism.
    To $f^* = H^*(f)$, Poincaré duality assigns the \enquote{Gysin morphism} $f_* : H^*(X)(d_X) \to H^{* - 2(d_X - d_Y)}(Y)(d_Y)$, adjoint to $f^*$ and satisfying the \emph{projection formula} $f_*(f^*(y) \cdot x) = y \cdot f_*(x)$.
    One also abusively denotes $f_*$ the homomorphisms obtained by the Gysin morphism after Tate twists.
    \item The trace map $\Tr_X$ is $p^X_*$, $p^X$ being the structural morphism $X \to \Spec k$.
    For each morphism $f : X \to Y$ as before, we have $\Tr_X = \Tr_Y \circ f_*$. From $p^{X \times Y} = p^X \times p^Y$ and the Künneth formula, we deduce $\Tr_{X \times Y} = \Tr_X \otimes \Tr_Y$.
    \item By the Künneth formula and duality, there are non graded canonical isomorphisms
    $$H(X \times Y)(d_X) \cong H(X)(d_X) \otimes H(Y) \cong \Hom_K(H(X), H(Y))$$
    given by $x \otimes y \mapsto (z \mapsto \langle z, x\rangle y)$, and more precisely there are canonical isomorphisms $u \mapsto \underline{u} = (z \mapsto \pr^{XY}_{Y*}(\pr^{XY*}_{X}(z) \cdot u))$:
    \begin{align*}
        H^{2d_X + i}(X \times Y)(d_X) &\cong \bigoplus_{j \geq 0} H^{2d_X - j}(X)(d_X) \otimes H^{j+i}(Y)\\
        &\cong \bigoplus_{j \geq 0} \Hom_K(H^j(X), H^{j+i}(Y))
    \end{align*}
    Moreover, for each $v \in H^{2d_Y + h}(Y \times Z)(d_Y)$ the element
    $$w = \pr^{XYZ}_{XZ*}(\pr^{XYZ*}_{XY}(u) \cdot \pr^{XYZ*}_{YZ}(v))$$
    satisfies $\underline{w} = \underline{v} \circ \underline{u} \in \bigoplus_{j \geq 0} \Hom_K(H^j(X), H^{j+i+h}(Z))$.
    \item Let $x \in H^i(X)$, $y \in H^j(Y)$ with $i = j$ mod $2$. We have
    $$\langle y', \underline{x \otimes y}(x')\rangle = \langle x', \underline{y \otimes x}(y')\rangle,$$
    so that we can identify $(-1)^i \underline{y \otimes x}$ with the Tate-twisted transpose of $\underline{x \otimes y}$.
    It follows that denoting $c_{XY}$ the switching map $X \times Y \to Y \times X$, we have
    $$\forall u \in H^{2d_X + 2r}(X \times Y)(d_X),\quad \underline{c_{XY}^*(u)} = \underline{u}^{\top}.$$
    \item Let $x \in H^{\text{even}}(X \times W)(d_X)$, $y \in H^{\text{even}}(Y \times Z)(d_Y)$, $v \in H(X \times Y)(d_X)$ and $w = \underline{c_{WY}^*(x \otimes y))}(v) \in H(W \times Z)(d_W)$. Then $\underline{w} = \underline{y} \circ \underline{v} \circ \underline{x}^{\top}$.
    \item $\gamma_X^0([X])$ is the unit of the algebra $H^*(X)$.
    \item Functoriality and compatibility of cycle classes with the \enquote{external} product imply compatibility with the \enquote{internal} product (intersection product in $\CH(X)$, cup-product in $H^*(X)$):
    $$\gamma_X^{r+s}(\alpha \cdot \beta) = \gamma_X^{r+s}(\Delta^*(\alpha \times \beta)) = \Delta^*(\gamma_X^r(\alpha) \otimes \gamma_X^s(\beta)) = \gamma_X^r(\alpha) \cdot \gamma_X^s(\beta).$$
    \item Let $g : Y \to X$ be a morphism in $\mathcal{P}(k)$ and suppose that $X$ has pure dimension $d_X$.
    Then with
    $$u = \gamma_{X \times Y}(\Gamma_g^{\top}) \in H^{2d_X}(X \times Y)(d_X),$$
    we have $\underline{u} = g^* \in \Hom_K(H^*(X), H^*(Y))$.
    If $Y$ has pure dimension $d_Y$ then with
    $$u^{\top} = \gamma_{X \times Y}(\Gamma_g) \in H^{2d_Y}(X \times Y)(d_X),$$
    we also have
    $$\underline{u^{\top}} = g_* \in \Hom_K(H^*(Y), H^{* - 2(d_X - d_Y)}(X)(d_X - d_Y)).$$
    \item Let $\beta \in \CH(X \times Y)$.
    Reusing the notation $\theta \in \CH(X) \mapsto \beta_*(\theta) = \pr^{XY}_{Y*}(\beta \cdot \pr^{XY*}_X(\theta)) \in \CH(Y)$ already introduced in \ref{MISSING REF 3.1.1}, we have $\underline{\gamma_{X \times Y}(\beta)} \circ \gamma_X = \gamma_Y \circ \beta_*$.
    When $\beta = \Gamma_f$ is the graph of a morphism $f : X \to Y$, we have $\underline{\gamma_{X \times Y}(\Gamma_f)} = f_*$, whence $f_*\gamma_X = \gamma_Y f_*$.
    \item $H^*(X \amalg Y)$ is canonically isomorphic to $H^*(X) \oplus H^*(Y)$.
    This lets us circumvent any equidimensionality assumption by letting
    $$H^{2d_X + i}(X \times Y)(d_X) = \bigoplus_j H^{2d_j + i} (X_j \times Y)(d_j)$$
    where $X_j$ is the component of $X$ having pure dimension $d_j$.
\end{itemize}

\subsection{Lefschetz trace formula.} Let $V = V^+ \oplus V^-$ and $W = W^+ \oplus W^-$ be two (finite-dimensional) super-vector spaces and let $\tilde{V}$ and $\tilde{W}$ be their respective $K$-duals.
Recall that the commutativity constraint $c_{VW}$ is the interchange of factors with the Koszul sign rule
$$V \otimes W \cong W \otimes V,\quad c_{VW}(v \otimes w) = (-1)^{\delta v \cdot \delta w} w \otimes v,$$
where $\delta v$ (resp. $\delta w$) denotes the parity of a homogeneous element $v$ (resp. $w$).

We identify $V \otimes \tilde{W}$ with the dual of $\tilde{V} \otimes W$ by the formula $\langle \tilde{v} \otimes w, v \otimes \tilde{w}\rangle = (-1)^{\delta v \cdot \delta w} \langle \tilde{v}, v\rangle\langle w, \tilde{w}\rangle$ (also we can assume that $\delta v = \delta \tilde{v}$ and $\delta w = \delta \tilde{w}$).

Moreover, we identify $\tilde{V} \otimes W$ with $\Hom(V, W)$ by $\tilde{v} \otimes w \mapsto (v' \mapsto \langle \tilde{v}, v'\rangle w)$ and similarly $\tilde{W} \otimes V$ is identified with $\Hom(W, V)$.

As such, the endomorphism $(\tilde{w} \otimes v) \circ (\tilde{v} \otimes w)$ of $V$ is given by $v' \mapsto \langle w, \tilde{w} \rangle \langle v', \tilde{v} \rangle v$, that is, $(\tilde{w} \otimes v) \circ (\tilde{v} \otimes w) = \langle w, \tilde{w} \rangle \tilde{v} \otimes v$.
The supertrace $s\tr(\tilde{v} \otimes v)$ is $\langle \tilde{v}, v\rangle$.

We deduce that $\langle \tilde{v} \otimes w, c_{\tilde{W}V}(\tilde{w} \otimes v)\rangle = s\tr((\tilde{w} \otimes v) \circ (\tilde{v} \otimes w))$, and by linearity that
$$\langle \phi, c_{\tilde{W}V}(\psi)\rangle = s\tr(\psi \circ \phi)$$
for each $\phi \in \Hom(V, W)$ and $\psi \in \Hom(W, V)$ of matching parities.

Apply this super-linear algebra formula to
$$V^+ = H^+(X),\quad V^- = H^-(X),\quad W^+ = H^+(Y),\quad W^- = H^-(Y),$$
and
$$\phi \in H^{2d_Y + i}(X \times Y)(d_Y + n),\quad \psi \in H^{2d_X - i}(Y \times X)(d_X - n),$$
where $H^+$ and $H^-$ respectively denote the even and odd parts of cohomology.
We also have $c_{\tilde{W}V}(\psi) = (c_{YX})_*(\psi) = (c_{XY})^*(\psi) = \psi^{\top}$.

We then obtain the \emph{Lefschetz trace formula}:
$$\boxed{\langle \phi, \psi^{\top} \rangle = \sum_{j = 0}^{2d_X} (-1)^j \tr(\psi \circ \phi | H^j(X)).}$$
In the case where $X = Y$, $i = n = 0$ and $\psi = \pi_X^j$, we have $(\pi_X^j)^{\top} = \pi_X^{2d_X - j}$, whence
$$\boxed{\tr(\phi | H^j(X)) = (-1)^j \langle \phi, \pi_X^{2d_X - j} \rangle.}$$

\subsection{Homological equivalence.} Let $H^*$ be a Weil cohomology with coefficients in $K$ and $F$ be a subring of $K$.
An element $\alpha \in \mathcal{Z}^r(X)_F$ is \emph{homologically equivalent to $0$} if $H^*(\alpha) = \gamma^r(\alpha) = 0$.
It follows from the above formula $\gamma_X^{r+s}(\alpha \cdot \beta) = \gamma_X^r(\alpha) \cdot \gamma_X^s(\beta)$ that this is an adequate equivalence relation, denoted $\sim_{\hom}$ or $\sim_H$.

Via $\gamma_X^r$, $\mathcal{Z}_{\hom}^r(X)_F$ is identified with an $F$-submodule (necessarily without torsion) of $H^{2r}(X)(r)$.
It thus has finite rank if $F = K$, but in general it is not known whether the quotient map $\mathcal{Z}_{\hom}^r(X)_F \otimes_{\Z} K \to \mathcal{Z}_{\hom}^r(X)_K$ is injective\footnote{if $K$ has positive characteristic $p$, this is false. There are counter-examples with $X$ a supersingular abelian surface over a field $k$ of characteristic $p$ and $r = 1$.}, and even whether $\mathcal{Z}_{\hom}^r(X)$ has finite rank.

It is immediate that $\sim_{\otimes\nil}$ is finer than $\sim_{\hom}$.
On the other hand, $\sim_{\hom}$ is finer than $\sim_{\num}$ (by multiplicativity of cycle classes, or using exercise \ref{MISSING REF 3.2.7.2}).

\subsubsection{Exercise.} Show that $\sim_{\hom}$ is coarser than $\sim_{\alg}$ without using \ref{MISSING REF 3.2.4.1}.
\end{document}