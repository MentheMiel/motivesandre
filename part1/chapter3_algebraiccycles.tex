\documentclass[../main.tex]{subfiles}

\begin{document}

In this second preliminary chapter, we present the basic tools of the theory of pure motives : algebraic cycles, adequate equivalences and Weil cohomologies.

\section{Algebraic cycles and adequate relations}

\subsection{} Let $k$ be a base field and denote $\mathcal{P}(k)$ the \emph{category of smooth projective schemes over $k$}, also sometimes called smooth projective $k$-varieties from now on.

For each $X \in \mathcal{P}(k)$ let $\mathcal{Z}^*(X)$ be the graded group of algebraic cycles on $X$, that is, the free abelian group generated by the integral closed subschemes $Z$ of $X$ and graded by codimension\footnote{we do not ask of $X$ that it is connected or even equidimensional, the codimension of an integral subscheme is still well-defined.}.
Denote by $[Z]$ the image of the subscheme $Z$ in $\mathcal{Z}^*(X)$ (and in its quotients).

For each commutative ring $F$, the elements of
$$\mathcal{Z}^r(X)_F = \mathcal{Z}^r(X) \otimes_{\Z} F$$
are called the \emph{algebraic cycles of codimension\footnote{we avoid the word \enquote{degree} to prevent any confusion with the degree of $0$-cycles defined in \ref{MISSING REF 3.2.6} or with the degree of the cycle class in cohomology, which is twice the codimension. Furthermore, it is also useful to consider the grading defined by dimension instead of codimension.} $r$ with coefficients in $F$}.

Intersection theory constructs a partially defined bilinear composition law on algebraic cycles: the intersection product, counting multiplicites, of two cycles whose components intersect properly (see \cite{se57}).
To extend this definition to all cycles, the classical approach consists in using an adequate equivalence relation, see \cite{sa58}.

\subsubsection{Definition.} An equivalence relation $\sim$ on algebraic cycles is said to be \emph{adequate} if it satisfies the following conditions for each $X, Y \in \mathcal{P}(k)$ :
\begin{enumerate}[label=\arabic*)]
    \item $\sim$ is compatible with the $F$-linear structure and with the grading,
    \item for each $\alpha, \beta \in \mathcal{Z}^*(X)_F$ there exists an $\alpha' \sim \alpha$ such that $\alpha'$ and $\beta$ intersect properly (so that the intersection product $\alpha' \cdot \beta$ is well-defined),
    \item for each $\alpha \in \mathcal{Z}^*(X)_F$ and each $\gamma \in \mathcal{Z}^*(X \times Y)_F$ intersecting properly $(\pr^{XY}_X)^{-1}(\alpha)$, we have $\alpha = 0 \implies \gamma_*(\alpha) \sim 0$ where
    $$\gamma_*(\alpha) = \pr^{XY}_Y(\gamma \cdot (\pr^{XY}_X)^{-1}(\alpha)).$$
\end{enumerate}

\subsubsection{Exercise.} Show that these conditions ensure that an \enquote{external} product $\alpha \times \beta$ (on $X \times Y$) is $\sim 0$ as soon as $\alpha \sim 0$ or $\beta \sim 0$.
\end{document}